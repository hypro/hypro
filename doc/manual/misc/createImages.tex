\documentclass[convert={density=300,size=1080x800,outext=.png}]{standalone}

%\usepackage{latexsym}
\usepackage{graphicx}
\usepackage{tikz}
\usepackage{pgfplots}
\usetikzlibrary{external}
\tikzexternalize[prefix=figures/]

\usepackage{color, colortbl}
\usepackage{rwthcolors}
\usepackage[siunitx]{circuitikz}
\usetikzlibrary{arrows,shadows,petri,automata,patterns}
\usetikzlibrary{decorations.pathmorphing}
\usetikzlibrary{decorations.markings}
\usetikzlibrary{calc}
\usetikzlibrary{intersections}

\tikzset{
	c/.style={every coordinate/.try}
}

\begin{document}

	\tikzsetnextfilename{boxInterval}
	\begin{tikzpicture} [>=latex]
		\draw[help lines] (-.5,-.5) grid (2.5,2.5);
		\draw[->] (-.5,0) -- node[below,pos=.9] {$x$} (2.5,0);
		\draw[->] (0,-.5) -- node[left,pos=.9] {$y$} (0,2.5);
		
		\draw[fill=blue100, draw=none, opacity=.3] (.5,-.5) rectangle (1.5,2.5);
		\draw[blue100] (.5,-.5) -- (.5,2.5);
		\draw[blue100] (1.5,-.5) -- (1.5,2.5);
		\node[] at(1,-.3) {$I_x$};
		
		\draw[fill=blue100, draw=none, opacity=.3] (-.5,.5) rectangle (2.5,1.8);
		\draw[blue100] (-.5,.5) -- (2.5,.5);
		\draw[blue100] (-.5,1.8) -- (2.5,1.8);
		\node[] at(-.3,1.15) {$I_y$};
		
		\draw[blue100, thick] (.5,.5) rectangle (1.5,1.8);
	\end{tikzpicture}
	
	\tikzsetnextfilename{boxMinMaxPoint}
	\begin{tikzpicture} [>=latex]
		\draw[help lines] (-.5,-.5) grid (2.5,2.5);
		\draw[->] (-.5,0) -- node[below,pos=.9] {$x$} (2.5,0);
		\draw[->] (0,-.5) -- node[left,pos=.9] {$y$} (0,2.5);
		
		\fill[fill=blue100] (.5,.5) circle [radius=2pt];
		\node[] at(.5,.25) {$\min$};
		
		\fill[fill=blue100] (1.5,1.8) circle [radius=2pt];
		\node[] at(1.5,2.05) {$\max$};
		
		\draw[blue100, thick, fill=blue100, fill opacity=.4] (.5,.5) rectangle (1.5,1.8);
	\end{tikzpicture}
	
	\tikzsetnextfilename{hPolytope}
	\begin{tikzpicture} [>=latex]
		\draw[help lines] (-.5,-.5) grid (2.5,2.5);
		\draw[->] (-.5,0) -- node[below,pos=.9] {$x$} (2.5,0);
		\draw[->] (0,-.5) -- node[left,pos=.9] {$y$} (0,2.5);
		
		\draw[fill=blue100, opacity=.3] (-.5,0) -- (1,2.5) -- (2.5,2.5) -- (2.5,-.5) -- (-.5,-.5) -- cycle;
		\draw[blue100] (-.5,0) coordinate(a_0) -- (1,2.5) coordinate(a_1);
		
		\draw[fill=blue100, opacity=.3] (-.5,1.5) -- (2.5,0) -- (2.5,-.5) -- (-.5,-.5) -- cycle;
		\draw[blue100] (-.5,1.5) coordinate(b_0) -- (2.5,0) coordinate(b_1);
		
		\draw[fill=blue100, opacity=.3] (-.5,.5) -- (2.5,.5) -- (2.5,2.5) -- (-.5,2.5) -- cycle;
		\draw[blue100] (-.5,.5) coordinate(c_0) -- (2.5,.5) coordinate(c_1);
		
		\coordinate (d_0) at (intersection of a_0--a_1 and b_0--b_1);
		\coordinate (d_1) at (intersection of a_0--a_1 and c_0--c_1);
		\coordinate (d_2) at (intersection of b_0--b_1 and c_0--c_1);
		
		\draw[blue100, thick] (d_0) -- (d_1) -- (d_2) -- cycle;
	\end{tikzpicture}

	\tikzsetnextfilename{vPolytope}
	\begin{tikzpicture} [>=latex]
		\draw[help lines] (-.5,-.5) grid (2.5,2.5);
		\draw[->] (-.5,0) -- node[below,pos=.9] {$x$} (2.5,0);
		\draw[->] (0,-.5) -- node[left,pos=.9] {$y$} (0,2.5);
		
		\draw[draw=none] (-.5,0) coordinate(a_0) -- (1,2.5) coordinate(a_1);
		\draw[draw=none] (-.5,1.5) coordinate(b_0) -- (2.5,0) coordinate(b_1);
		\draw[draw=none] (-.5,.5) coordinate(c_0) -- (2.5,.5) coordinate(c_1);
		
		\coordinate (d_0) at (intersection of a_0--a_1 and b_0--b_1);
		\coordinate (d_1) at (intersection of a_0--a_1 and c_0--c_1);
		\coordinate (d_2) at (intersection of b_0--b_1 and c_0--c_1);
		
		\fill[fill=blue100] (d_0) circle [radius=2pt];
		\fill[fill=blue100] (d_1) circle [radius=2pt];
		\fill[fill=blue100] (d_2) circle [radius=2pt];
		
		\draw[blue100, thick, fill=blue100, fill opacity=.5] (d_0) -- (d_1) -- (d_2) -- cycle;
	\end{tikzpicture}
	
	\tikzsetnextfilename{zonotope}
	\begin{tikzpicture} [>=latex]
		\draw[help lines] (-.5,-.5) grid (2.5,2.5);
		\draw[->] (-.5,0) -- node[below,pos=.9] {$x$} (2.5,0);
		\draw[->] (0,-.5) -- node[left,pos=.9] {$y$} (0,2.5);
		
		\draw[blue100, ->] (0,0)  -- node[black, left,pos=0.9]{$c$} (1,1);
		
		\draw[blue100, ->] (1,1)  -- node[black, left,pos=0.5]{$g_0$} (1.2,1.8);
		\draw[blue100, ->] (1,1)  -- node[black, below,pos=0.5]{$g_1$} (2,1.2);
		
		% for testing
		\draw[blue100, ->, dotted] (1,1)  -- (0.8,0.2);
		\draw[blue100, ->, dotted] (1,1)  -- (0,0.8);
		
		\draw[blue100, fill=blue100, fill opacity=.5, thick] (2.2,2) -- (1.8,0.4) -- (-.2, 0) -- (0.2,1.6) -- cycle;
	\end{tikzpicture}
	
	\tikzsetnextfilename{supportFunction}
	\begin{tikzpicture} [>=latex]
		\draw[help lines] (-.5,-.5) grid (2.5,2.5);
		\draw[->] (-.5,0) -- node[below,pos=.9] {$x$} (2.5,0);
		\draw[->] (0,-.5) -- node[left,pos=.9] {$y$} (0,2.5);
		
		\draw [black] plot [smooth cycle] coordinates {(.5,1) (1,1.8) (1.8,1) (1,0.5) };
		
		\draw[blue100, ->] (1,1) -- (1.5,1);
		\draw[blue100, ->] (1,1) -- (1.2,1.5);
		\draw[blue100, ->] (1,1) -- (.5,1);
		\draw[blue100, ->] (1,1) -- (.7,.7);
		\draw[blue100, ->] (1,1) -- (1.2,.5);
		
		\draw[blue100] (1.8,-.5) coordinate(a0) --  (1.8,2.5) coordinate(a1);
		\draw[blue100] (.5,-.5) coordinate(b0) -- (.5,2.5) coordinate(b1);
		\draw[blue100] (-.5,1.85) coordinate(c0) -- (1.85,-.5) coordinate(c1);
		\draw[blue100] (-.5,2.42) coordinate(d0) -- (2.5,1.22) coordinate(d1);
		\draw[blue100] (-.5,-.12) coordinate(e0) -- (2.5,1.02) coordinate(e1);
		
		\coordinate (f0) at (intersection of a0--a1 and d0--d1);
		\coordinate (f1) at (intersection of d0--d1 and b0--b1);
		\coordinate (f2) at (intersection of b0--b1 and c0--c1);
		\coordinate (f3) at (intersection of c0--c1 and e0--e1);
		\coordinate (f4) at (intersection of e0--e1 and a0--a1);
		
		\draw[thick, blue100, fill=blue100, fill opacity=.5] (f0) -- (f1) -- (f2) -- (f3) -- (f4) -- cycle;
	\end{tikzpicture}

\end{document}
